\documentclass{article}
\usepackage{caption} % Caption mynda
\usepackage{hyperref}
\usepackage{graphicx, float} % Insetning mynda
\usepackage{keystroke} % Veit ekki
\usepackage{pgfplots} % Veit ekki
\usepackage{enumerate} % Numbered-bullet-points
\usepackage{amsmath} % E.g. for formatting equations without numbers (equation*)
\usepackage{listings} % Insert code to document
\pgfplotsset{compat=1.17}
% \usepackage{showlabels}
\usepackage[T1]{fontenc} % Encoding
\usepackage[utf8]{inputenc} % Encoding
% \usepackage[icelandic]{babel} %Encoding

\usepackage{geometry}
\geometry{
    paper=a4paper, % letterpaper lika til
	top=2.5cm, % Top margin
	bottom=2cm, % Bottom margin
	left=3.5cm, % Left margin
	right=3.4cm, % Right margin
	headheight=0.75cm, % Header height
	footskip=1cm, % Space from the bottom margin to the baseline of the footer
	headsep=0.75cm, % Space from the top margin to the baseline of the header
	%showframe, % Uncomment to show how the type block is set on the page
}
    
% \usepgfplotslibrary{units}
\RequirePackage{fancyhdr}
\linespread{1.3} % Línubil
\fontfamily{pbch}\selectfont
\usepackage{color}

\definecolor{mygreen}{rgb}{0,0.6,0}
\definecolor{mygray}{rgb}{0.5,0.5,0.5}
\definecolor{mymauve}{rgb}{0.58,0,0.82}
\lstset{
showstringspaces=false,
backgroundcolor=\color{white},
basicstyle=\footnotesize,
breaklines=true,
captionpos=b,
commentstyle=\color{mygreen},
keywordstyle=\color{blue},
stringstyle=\color{mymauve},
}

% \usepackage{zref}
% \usepackage{biblatex} %Imports biblatex package
%\usepackage[backend=biber]{biblatex}
%\addbibresource{references.bib} %Import the bibliography file
\begin{document}

\today \par
\vspace{.5cm}
\noindent Háskólinn í Reykjavík, Embedded Systems Programming, \textbf{Project 4} \par
\noindent \textbf{Eyþór Mikael Eyþórsson}, \texttt{eythore19@ru.is}\par
\noindent \textbf{Vilhjálmur Páll Thorarensen}, \texttt{vilhjalmurt19@ru.is}\par
%\noindent Instructors: \textbf{TEACHER} \par

\section*{Part 0}
% Wire RPi input and output pins to the motor encoder outputs and (optionally) an external LED. There isn't a pin header on the Raspberry Pi Zero 2 W.  In V207, you can obtain a header and solder it onto the board.  The pins should be oriented upwards if you are using a breakout board. Be sure there are no solder bridges that short-circuit between solder pads.
The RPi input and output pins were wired to the motor encoder and external LED 
in the manner shown in Figure~\ref{fig:rpi-encoder-connection}
\begin{figure}[h]
	\begin{center}
		\includegraphics[width=0.45\textwidth]{~/Pictures/big-chungus.png}
	\end{center}
	\caption{The motor encoder connected to the Raspberry
		Pi}\label{fig:rpi-encoder-connection}
\end{figure}

\section*{Part 1}

The state of the encoder was written dynamically to the LED pin using \begin{enumerate}
    \item Gpiod commands from the shell script in Listing~\ref{lst:shell-script}
    \item Gpiod from the C++ application in Listing~\ref{lst:cpp-polling}, polling the pin status using a timed read
    \item Gpiod from the C++ application in Listing~\ref{lst:cpp-events} making use of gpiod events 
    \item The kernel module in Listing~\ref{lst:kernel-module} using interrupts
\end{enumerate} and the response time and CPU load were measured using an oscilloscope and
the \texttt{top} command, respectively. The results were compared in
Table~\ref{tab:part1-results} 

\subsection*{Bash script}\label{sub:Bash script} % (fold)
\begin{lstlisting}[language=bash,caption={Shell script},label=lst:shell-script]
#!/bin/bash
echo anus licker
\end{lstlisting}

% subsection Bash script (end)

\subsection*{Polling in C++}\label{sub:Polling in C++} % (fold)
\begin{lstlisting}[language=c++,caption={C++ application polling pin status using gpiod},label=lst:cpp-polling]
#!/bin/bash
echo anus licker
\end{lstlisting}

% subsection Polling in C++ (end)

\subsection*{Events in C++}\label{sub:Events in C++} % (fold)
\begin{lstlisting}[language=c++,caption={C++ application reading pins using
events},label=lst:cpp-events]
#!/bin/bash
echo anus licker
\end{lstlisting}

% subsection Events in C++ (end)

\subsection*{Kernel module}\label{sub:Kernel module} % (fold)
\begin{lstlisting}[language=c,caption={Kernel module},label=lst:kernel-module]
#!/bin/bash
echo anus licker
\end{lstlisting}

% subsection Kernel module (end)

\subsection*{Conclusion}


\begin{itemize}
    \item Which mechanism would suffice for counting encoder pulses?
    \item What is the associated CPU load (reported by top)?
    \item Do the results change when the CPU is fully loaded (by some other process)?
    \item Adjust the process priority in an attempt to improve response under CPU load conditions.
\end{itemize}

\begin{table}[h]
    \caption{Results of Part 1, showing CPU load and response time for various methods of
    performing IO operations}\label{tab:part1-results}
    \begin{center}
        \begin{tabular}[c]{l|c|c}
            \hline
            \multicolumn{1}{c|}{\textbf{Method description}} & 
            \multicolumn{1}{c|}{\textbf{Response time[ms]}} & 
            \multicolumn{1}{c}{\textbf{CPU load[\%]}} \\
            \hline
            homosex & 69 & 99 \\
            \hline
        \end{tabular}
    \end{center}
\end{table}



\section*{Part 2}

%\newpage
%\printbibliography

\end{document}
